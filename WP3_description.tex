\subsection{Work Package 3}

\begin{table}[hbpt]\centering
	\begin{tabular}{|p{0.35\linewidth}|p{0.06\linewidth}|p{0.06\linewidth}|p{0.06\linewidth}|
         p{0.06\linewidth}|p{0.06\linewidth}|p{0.06\linewidth}|p{0.06\linewidth}|}\hline
		 Work package number& WP3 &
		 \multicolumn{4}{|c|}{Start date or starting event:}{}&
		 \multicolumn{2}{|c|}{                        4 }{}\\\hline
		 Work package title&\multicolumn{7}{|c|}{\gls{NLP} for Requirements Analysis}{}\\\hline
		 Activity Type&\multicolumn{7}{|c|}{ RTD}{}\\\hline
		 Participant number & 4 & 7 & 8 & 9 & ~ & ~ & ~ \\\hline
		 Person-months per participant: & 20 & 3 & 5 & 8 & ~ & ~ & ~ \\\hline
	\end{tabular}
\end{table}

\subsubsection{Objectives}
\begin{itemize}
	\item O3.1 Create the \gls{NLP}: The \gls{NLP} should be able to transform the natural language requirements into a formalized description of the requirements. The architecture should match the problem at hand. It has to be fast and easily integrable into multiple Requirement Engineering Tools such as DOORS or Polarion.
	
	The \gls{NLP} should be trained on historical data and on small batches to learn basic relationships. The \gls{NLP} should be able to reproduce the same or better formal definition as the ones it was trained on.
	
	The \gls{NLP} should be validated on the companies real life historical project requirements. The results should be reported and sent back for perfection.
	
	\item O3.2 Present Research Work: The research results should be presented and summarized at the end of the work. 
\end{itemize}

\subsubsection{Description of work}
\begin{itemize}
	\item O3.1 Create the \gls{NLP}:
	\begin{itemize}
		\item T3.1.1: The \gls{TUB} team should define the architecture for the \gls{NLP}. The architecture should be adequate for the task at hand.
		\item T3.1.2: The architecture has to be documented and presented to the stakeholders until M1.
		\item T3.1.3: The \gls{TUB} team should train the \gls{NLP} for the task at hand. All training data should be available for all stakeholders and the training data should be accepted by all stakeholders. The training data can be changed during the project, but it has to be always accepted by all stakeholders.
		\item T3.1.4: The companies have to validate the trained \gls{NLP} on their historical real life project requirements. The companies has to be involved as soon as as possible and the validation should be repeated as often as possible. The results should be documented at a central location accessible by all stakeholders. The results should be taken into consideration for any change in the architecture or in the training data.  
	\end{itemize}
	\item O3.2 Present Research Work: The work should be summarized and be published as an article until M1. 
\end{itemize}
\subsubsection{Deliverables}
\begin{itemize}
	\item D3.1 First \gls{NLP} Prototype : The first prototype should be delivered at the end of the 7th months. It has to be able to run without crash and fulfill basic functionalities.
	\item D3.2 Final \gls{NLP} Prototype : The final prototype should be delivered until M1 and it has to fulfill all functionalities.
	\item D3.3 \gls{NLP} Project Paper : The project paper should be delivered until M1 and should contain the results of WP3.
\end{itemize}