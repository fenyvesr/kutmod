\newglossaryentry{nlpg}{name       ={NLP},
						description=
						{
							Natural language processing (NLP) is a subfield of linguistics, computer science, information engineering, and artificial intelligence concerned with the interactions between computers and human (natural) languages, in particular how to program computers to process and analyze large amounts of natural language data
						}
					   }

\newglossaryentry{srsg}{name       ={SRS},
						description=
						{
							A software requirements specifications (SRS) is a description of a software system to be developed. It is modeled after business requirements specification (CONOPS), also known as a stakeholder requirements specification (StRS). The software requirements specification lays out functional and non-functional requirements, and it may include a set of use cases that describe user interactions that the software must provide to the user for perfect interaction.
							Software requirements specification establishes the basis for an agreement between customers and contractors or suppliers on how the software product should function (in a market-driven project, these roles may be played by the marketing and development divisions). Software requirements specification is a rigorous assessment of requirements before the more specific system design stages, and its goal is to reduce later redesign. It should also provide a realistic basis for estimating product costs, risks, and schedules. Used appropriately, software requirements specifications can help prevent software project failure.
							The software requirements specification document lists sufficient and necessary requirements for the project development. To derive the requirements, the developer needs to have clear and thorough understanding of the products under development. This is achieved through detailed and continuous communications with the project team and customer throughout the software development process.
							The SRS may be one of a contract's deliverable data item descriptions or have other forms of organizationally-mandated content
						}
					   }

\newglossaryentry{sdlcg}{name      ={SDLC},
						 description=
						 {
							The Software Development Life Cycle (SDLC), also referred to as the application development life-cycle, is a process for planning, creating, testing, and deploying an information system. There are usually six stages in this cycle: requirement analysis, design, development and testing, implementation, documentation, and evaluation.
						 }
                        }

\newglossaryentry{postg}{name       ={POST},
						description=
						{
							In corpus linguistics, Part-of-Speech Tagging (POST), also called grammatical tagging or word-category disambiguation, is the process of marking up a word in a text (corpus) as corresponding to a particular part of speech, based on both its definition and its context (i.e. its relationship with adjacent and related words in a phrase, sentence, or paragraph).
							Once performed by hand, POST is now done in the context of computational linguistics, using algorithms which associate discrete terms, as well as hidden parts of speech, in accordance with a set of descriptive tags. POST algorithms fall into two distinctive groups: rule-based and stochastic.
						}
					}
				
\newglossaryentry{posg}{name       ={POS},
						description=
						{
							In traditional grammar, a Part of Speech (POS) is a category of of lexical items that have similar grammatical properties. Words that are assigned to the same POS generally display similar syntactic behavior and they play similar roles within the grammatical structure of sentences.
							Commonly listed English parts of speech are noun, verb, adjective, adverb, pronoun, preposition, conjunction, interjection, and sometimes numeral, article, or determiner. Other Indo-European languages also have essentially all these word classes. One exception to this generalization is that most Slavic languages as well as Latin and Sanskrit do not have articles. Beyond the Indo-European family, such other European languages as Hungarian and Finnish, both of which belong to the Uralic family, completely lack prepositions or have only very few of them; rather, they have postpositions.
						}
}

\newglossaryentry{aspiceg}{name       ={ASPICE},
	description=
	{
		Automotive \gls{SPICE} (ASPICE) is a standard made by german car makers. It provides rough guidelines to improve your software development processes and to assess suppliers. This means that practically the whole european automotive industry must follow ASPICE and if you want to know how software development works there it provides a good broad overview.
		Automative \gls{SPICE} is derived from the generic \gls{SPICE} (ISO/IEC 15504) standard. While there are other instances, ASPICE seems to be the only one which got any traction. You can find a lot of ASPICE material on the internet unrestricted by paywalls.
		It builds on the V-Model which means for every process from requirements to source code there is a corresponding test.
	}
}

\newglossaryentry{spiceg}{name       ={SPICE},
						  description=
						  {
							ISO/IEC 15504 Information technology – Process assessment, also termed Software Process Improvement and Capability Determination (SPICE), is a set of technical standards documents for the computer software development process and related business management functions. It is one of the joint International Organization for Standardization (ISO) and International Electrotechnical Commission (IEC) standards, which was developed by the ISO and IEC joint subcommittee, ISO/IEC JTC 1/SC 7.
							ISO/IEC 15504 was initially derived from process lifecycle standard ISO/IEC 12207 and from maturity models like Bootstrap, Trillium and the Capability Maturity Model (CMM).
							ISO/IEC 15504 has been revised by: ISO/IEC 33001:2015 Information technology – Process assessment – Concepts and terminology as of March, 2015 and is no longer available at ISO.
						}
					}
				
\newglossaryentry{misrag}{name       ={MISRA},
						description=
						{
							Motor Industry Software Reliability Association (MISRA) is an organization that produces guidelines for the software developed for electronic components used in the automotive industry. It is a collaboration between vehicle manufacturers, component suppliers and engineering consultancies.
						}
}