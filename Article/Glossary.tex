\newglossaryentry{nlpg}{name={NLP},
						description=
						{
							Natural language processing (NLP) is a subfield of linguistics, computer science, information engineering, and artificial intelligence concerned with the interactions between computers and human (natural) languages, in particular how to program computers to process and analyze large amounts of natural language data
						}
					   }

\newglossaryentry{srsg}{name={SRS},
						description=
						{
							A software requirements specifications (SRS) is a description of a software system to be developed. It is modeled after business requirements specification (CONOPS), also known as a stakeholder requirements specification (StRS). The software requirements specification lays out functional and non-functional requirements, and it may include a set of use cases that describe user interactions that the software must provide to the user for perfect interaction.
							Software requirements specification establishes the basis for an agreement between customers and contractors or suppliers on how the software product should function (in a market-driven project, these roles may be played by the marketing and development divisions). Software requirements specification is a rigorous assessment of requirements before the more specific system design stages, and its goal is to reduce later redesign. It should also provide a realistic basis for estimating product costs, risks, and schedules. Used appropriately, software requirements specifications can help prevent software project failure.
							The software requirements specification document lists sufficient and necessary requirements for the project development. To derive the requirements, the developer needs to have clear and thorough understanding of the products under development. This is achieved through detailed and continuous communications with the project team and customer throughout the software development process.
							The SRS may be one of a contract's deliverable data item descriptions or have other forms of organizationally-mandated content
						}
					   }