\section{Results}
\begin{minipage}{\linewidth}
	\begin{multicols*}{2}
		In this paper, we contributed an approach to extract functional requirements from free text documents for supporting domain analysis. We proposed an extended variability model to help the detailed and integrated expression of product line functional requirements. And then we automatically convert dependency relations of words in a sentence into cases to support the modeling of the variability of product line functional requirements. These pre-processed requirements could be fed to a recurrent neural network which could generate the semantic meaning og the natural language functional requirements. The performance of our \gls{RNN} can be seen in the following table.
	\end{multicols*}
		\begin{tabular}{|c|c|c|c|c|}\hline
			Metrics& LSTM Metics 30 lags &\vtop{\hbox{\strut LSTM Metrics}\hbox{\strut Optimal Time Lags}} &\vtop{\hbox{\strut Extra Tree}\hbox{\strut Model Metrics}} &Error Reduction (\%)\\\hline
			RMSE & 353.38 & 341.40 & 428.1 &20.3\\
			CV(RMSE)&0.643& 0.622& 0.78& 20.3\\
			MAE&263.14&249.53&292.49&14.9\\\hline
		\end{tabular}
	\begin{multicols*}{2}
		Another result is that we were able to define the full formal semantics of \gls{MISRA} C. We covered all cases and rules defined by \gls{MISRA} C. From this result we were able to analyze the formal semantics of various automotive C source codes. During our analysis we were able to spot multiple ambiguities and design faults. The semantic analysis itself can help improve the code quality and help avoid misunderstandings during the development procedure.
		
		At the end the semantics of the requirements and the semantics of the source code were compared against each other. The verifier is rigorously proven to be correct. It can check whether the given source is fulfilling the given functional requirements. If the verifier finds the source code incorrect than in some cases it can suggest fixes. The fixes are proved to be correct and we defined goodness measures for ranking these plausible fixes.
		
		All together we were able to build a development environment which rigorously supports the software development from the beginning to the end. All steps are mathematically proven to be correct and our result is a state of the art solution for a real life problem. During the process we overcome many obstacles and improved many algorithms and methodologies.
	\end{multicols*}
\end{minipage}
